\documentclass[a4paper, 12pt, oneside]{article}
\usepackage[utf8]{inputenc}
\usepackage[english,russian]{babel}
\usepackage[unicode=true] {hyperref}

\begin{document}

%\maketitle
\begin{center}
\section*{Индивидуальное задание №~1. Целочисленная арифметика. \\ Одномерные массивы}
\end{center}

Разработать программу, которая вводит одномерный массив $A$, состоящий из $N$ элементов (значение $N$ вводится при выполненпии программы),после чего формирует из элементов массива $A$ новый массив $B$ по правилам, указанным в варианте, и выводит его. Память под массивы может выделяться статически, на стеке, автоматичеси по выбору разработчика с учетом требований к оценке работы.

При решении задачи необходимо использовать подпрограммы для реализации ввода, вывода и формирования нового массива массива. Допустимы (при необходимости) дополнительные подпрограммы.

Максимальное количество элементов в массиве не должно превышать 10 (ограничение обуславливается вводом данных с клавиатуры). При этом необходимо обрабатывать некорректные значения как для нижней, так и для верхней границ массивов в зависимости от условия задачи.

\begin{enumerate}
    \item Сформировать массив $B$ из положительных элементов массива $А$.
    \item Сформировать массив $B$ только из тех элементов массива $А$, которые не совпадают с его первым и последним элементами.
    \item Сформировать массив $B$ из сумм соседних элементов $A$ по следующим правилам: $B_0 =A_0 +A_1, B_1 = A_1 + A_2, ...$
    \item Массив $B$ из массива $A$ формируется по следующим правилам:
    \begin{itemize}
        \item $B_i =  1$, если $A_i > 0$,
        \item $B_i = -1$, если $A_i < 0$,
        \item $B_i =  0$, если $A_i = 0$.
    \end{itemize}
    \item Сформировать массив $B$, состоящий из элементов массива $А$, значение которых не совпадает с введённым числом $X$.
    \item Сформировать массив $B$, состоящи из элементов массива $A$, значения которых кратны введенному числу $X$.
    \item Сформировать массив $B$ из индексов положительных элементов массива $A$.
    \item Сформировать массив B по следующим правилам:
    \begin{itemize}
        \item если $A_i > 5$, то увеличить элемент на 5,
        \item если $A_i < -5$, то уменьшить на 5,
        \item остальныое обнулить.
    \end{itemize}
    \item Сформировать массив $B$ из нечётных элементов массива $A$.
    \item Сформировать массив $B$ из отрицательных элементов массива $A$, расположенных обратном порядке.
    \item Сформировать массив $B$ из элементов $A$, расположенных в обратном порядке, исключая первый положительный элемент.
    \item Сформировать массив $B$ из элементов массива $A$, исключив первый положительный и последний отрицательный элементы.
    \item Сформировать массив $B$ из элементов массива $A$, за исключением элементов, значения которых совпадают с минимальным элементом массива A.
    \item Сформировать массив $B$ из элементов массива $A$ заменой всех отрицательных значений на максимум из массива $A$.
    \item Сформировать массив $B$ из элементов массива $A$ заменой всех нулевых элементов значением минимального элемента.
    \item Сформировать массив $B$ из элементов массива $A$, заменой значений, которые больше среднего арифметического, значением среднего арифметического массива A.
    \item Сформировать массив $B$ из элементов массива $A$, расположенных после последнего положительного элемента.
    \item Сформировать массив $B$ из элементов массива $A$ уменьшением всех элементов до первого положительного на 5.
    \item Сформировать массив $B$ из элементов массива $A$ заменой нулевых элементов, предшествующих первому отрицательному, единицей.
    \item Сформировать массив $B$ из элементов массива $A$ перестановкой местами минимального и первого элементов.
    \item Сформировать отсортированный по возрастанию массив $B$ из элементов массива $A$.
    \item Сформировать отсортированный по убыванию массив $B$ из элементов массива $A$.
    \item Сформировать массив $B$, элементы которого являются  числами Фибоначчи от соответствующих элементов массива $A$. В случае переполнения в соответствующие места записывать нулевые значения.
    \item Сформировать массив $B$ из элементов массива $A$ поменяв местами элементы, стоящие на чётных и нечётных местах: \\ $A_0 \leftrightarrow A_1; A_2 \leftrightarrow A_3 ... $
    \item Сформировать массив $B$ из элементов массива $A$ заменив все положительные числа значением 2, а отрицательные — увеличить на 5.
    \item Сформировать массив $B$ из сумм трех соседних элементов массива $A$, где $B_i = A_{i-1}+A_{i}+A_{i+1}$. Если элементов в массиве А менее трёх, то заполнить массив В нулями.
    \item Сформировать массив $B$ из элементов массива $A$. Элементы массива $А$, оканчивающиеся цифрой 4, уменьшить вдвое.
    \item Сформировать массив $B$ из элементов массива $A$, которые образуют неубывающую последовательность. Неубывающей последовательностью считать элементы идущие подряд, которые равны между собой или каждый последующий больше предыдущего.
    \item Сформировать массив $B$ из сумм соседних элементов $A$ по следующему правилу: $B_0=A_0, B_1=A_0 + A_1, ..., B_m=A_0 + ... + A_m$, где $m$ – либо номер первого четного отрицательного элемента массива $А$, либо номер последнего элемента, если в массиве $А$ нет отрицательных элементов.
    \item Сформировать массив $B$ из элементов массива $A$, которые больше, чем элементы, стоящие перед ними.
    \item Сформировать массив $B$ из элементов массива $A$ в следующем порядке: сначала заполняем массив $В$ числами, стоящими на нечетных местах, а затем --- стоящими на четных местах в массиве $А$.
    \item Сформировать массив $B$ из элементов массива $A$, которые меньше суммы элементов, расположенных на четных местах.
    \item Сформировать массив $B$ из элементов массива $A$, путем записи разности между двумя соседним элементами.
    \item Сформировать массив $B$ из элементов массива $A$ заменив элементы на четных местах суммой всех положительных элементов, а элементы на нечетных местах суммой отрицательных элементов.
    \item Сформировать массив $B$ из элементов массива $A$ сгруппировов положительные элементы массива $А$ в начале массива $В$, нулевые в середине, а отрицательные — в конце.
    \item Сформировать массив $B$ из элементов массива $A$ сгруппировав элементы с четными индексами в начале массива $В$ , а элементы с нечетными индексами сгруппировать в конце массива В.
    \item Сформировать массив $B$ из элементов массива $A$ в следующем порядке: элементы с индексами $i \le (N + 1)/2$ переместить на позиции с четными индексами массива $В$ с сохранением их исходного порядка относительно друг друга, а оставшиеся элементы ($i > (N + 1)/2$) разместить на позициях с нечетными индексами массива $В$ также с сохранением их исходного порядка.
    \item Сформировать массив $B$ из элементов массива $A$, которые одновременно имеют четные и отрицательные значения.
    \item Сформировать массив B, элементы которого являются площадью квадратов со сторонами указанные в массиве A. Операцию умножения реализовать через сложение и/или сдвиги. При переполнении в массивв $B$ записывать нули.
    \item Сформировать массив $B$ из сумм соседних элементов $A$ по следующему правилу: $B_0=A_0, B_1=A_0+A_1, ..., B_m=A_0+ ... +A_m$, где $m$ – номер первого элемента массива А большего среднего арифметического этого массива. При переполнении записывать нули.
    \item Сформировать массив $B$ из индексов элементов массива $A$. Порядок следования индексов в массиве $B$ позволяет выводить элементы массива $A$ по возрастанию.
    \item Сформировать массив $B$ из индексов элементов массива $A$. Порядок следования индексов в массиве $B$ позволяет выводить элементы массива $A$ по убыванию.
    \item Сформировать массив $B$, элементы которого являются  факториалами от соответствующих элементов массива $A$. В случае переполнения в соответствующие места записывать нулевые значения.
\end{enumerate}

\end{document}
