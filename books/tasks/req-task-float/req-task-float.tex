\documentclass[a4paper, 12pt, oneside]{article}
\usepackage[utf8]{inputenc}
\usepackage[english,russian]{babel}
\usepackage[unicode=true] {hyperref}

\begin{document}

%\maketitle
\begin{center}
\section*{Требование к выполнению и оценка задания~№\,2}
\end{center}

Ниже представлены требования, которым должно удовлетворять задание по разработке программы на ассемблере для получения соответствующей оценки.

\textbf{Следует отметить, что необходимо представить только одну модифицированную ассемблерную программу, разработанную на целевую оценку}.

Программа должна запускаться и выполняться в симуляторе RARS.

\textbf{При разработке программы использовать числа с плавающей точкой двойной точности (double)}.

\textbf{\textit{Для обеспечения лучшей переносимости кода рекомендуется при работе с текстовыми сообщениями использовать английский язык или транслит. То есть, пользоваться только кодировкой ASCII.}}

Сформированный отчет должен содержать:
\begin{itemize}
    \item фамилию, имя, отчество студента;
    \item номер группы;
    \item номер варианта задания;
    \item условие задачи;
    \item описание метода решения задачи;
    \item ссылки источник или источники информации с описанием метода решения задачи;
    \item описание тестовых прогонов с представлением информации о результатах тестирования.
\end{itemize}
При невыполнении хотя бы одного из требований оценка снижается.

\subsection*{4--5 баллов}

\begin{itemize}
    \item Приведено решение задачи на ассемблере. Ввод данных осуществляется с клавиатуры. Вывод данных осуществляется на дисплей.
    \item В программе должны присутствовать комментарии, поясняющие выполняемые действия.
    \item Допускается использование требуемых подпрограмм без параметров и локальных переменных.
    \item В отчете должно быть представлено полное тестовое покрытие. Приведены результаты тестовых прогонов. Например, с использованием скриншотов.
\end{itemize}
При невыполнении хотя бы одного из требований оценка снижается.


\subsection*{6--7 баллов}

\textbf{При разработке программы на данную оценку необходимо учитывать все требования, предъявляемые на предшествующую оценку.}

\begin{itemize}
    \item В программе необходимо использовать подпрограммы с передачей аргументов через параметры, что обеспечивает их повторное использование с различными входными аргументами. При нехватке регистров, используемых для передачи параметров, оставшиеся параметры передавать через стек.
    \item Внутри подпрограмм необходимо использовать локальные переменные. При нехватке временных регистров обеспечить сохранение данных на стеке в соответствии с соглашениями, принятыми для процессора.
    \item В местах вызова функции добавить комментарии, описывающие передачу фактических параметров и перенос возвращаемого результата. При этом необходимо отметить, какая переменная или результат какого выражения соответствует тому или иному фактическому параметру.
    \item Информацию о проведенных изменениях отобразить в отчете наряду с информацией, необходимой на предыдущую оценку.
\end{itemize}
При невыполнении хотя бы одного из требований оценка снижается.

\subsection*{8 баллов}

\textbf{При разработке программы на данную оценку необходимо учитывать все требования, предъявляемые на предшествующие оценки.}

\begin{itemize}
    \item Разработанные подпрограммы должны поддерживать многократное использование с различными наборами исходных данных, включая возможность подключения различных исходных и результирующих массивов.
    \item Реализовать автоматизированное тестирование за счет создания \textbf{дополнительной тестовой программы}, осуществляющей прогон подпрограмм, осуществляющих вычисления для различных тестовых данных (вместо их ввода). Осуществить прогон тестов обеспечивающих покрытие различных ситуаций.
    \item Для дополнительной проверки корректности вычислений осуществить аналогичные тестовые прогоны с использованием существующих библиотек и одного из языков программирования высокого уровня по выбору: C, C++, Python.
    \item Добавить информацию о проведенных изменениях в отчет.
\end{itemize}
При невыполнении хотя бы одного из требований оценка снижается.

\subsection*{9 баллов}

\textbf{При разработке программы на данную оценку необходимо учитывать все требования, предъявляемые на предшествующие оценки.}

\begin{itemize}
    \item Добавить в программу использование макросов для реализации ввода и вывода данных. Макросы должны поддерживать повторное использование с различными массивами и другими параметрами.
    \item Допускается обертыванием макросами уже разработанных подпрограмм.
\end{itemize}
При невыполнении хотя бы одного из требований оценка снижается.

\subsection*{10 баллов}

\textbf{При разработке программы на данную оценку необходимо учитывать все требования, предъявляемые на предшествующие оценки.}

\begin{itemize}
    \item Программа должна быть разбита на несколько единиц компиляции. При этом подпрограммы ввода--вывода должны составлять унифицированные модули, используемые повторно как в программе, осуществляющей ввод исходных данных, так и в программе, осуществляющей тестовое покрытие.
    \item Макросы должны быть выделены в отдельную автономную библиотеку
    \item Расширить отчет, дополнив его новыми данными.
\end{itemize}
При невыполнении хотя бы одного из требований оценка снижается.

\end{document}
