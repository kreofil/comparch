\chapter{Эмулятор RARS процессора RISC-V}

RARS --- это интегрированная среда разработки, отладки и выполнения программ процессора RISC-V в режиме симуляции. Она написана на языке программирования Java, что обеспечивает переносимость между компьютерами с различной архитектурой, использующих разнообразные операционные системы. Распространяется в виде исполняемого файла JAR. Среда включает:
\begin{itemize}
    \item текстовый редактор, обеспечивающий написание программ;
    \item ассемблер, позволяющий получить исполняемый код путем сборки одного или нескольких модулей;
    \item эмулятор, осуществляющий моделирование процессора RISC-V и позволяющий выполнять программы, разработанные на языке ассемблера;
    \item отладчик, работающий как в пошаговом режиме, так и использующий точки останова;
    \item средства, обеспечивающие отображение информации о состоянии регистров моделируемого процессора и его оперативной памяти;
    \item модели внешних устройств, позволяющие имитировать взаимодействие с внешней средой в различных режимах.
\end{itemize}
Помимо этого компоненты, входящие в среду разработки можно запускать из командной строки.
\debate[Примечание]{Остальне будет дописываться по мере развития методы...}

\section{Использование RARS через интегрированную среду разработки (IDE)}

IDE вызывается, когда RARS запускается без аргументов команды, например: \verb|java -jar rars.jar|. Его также можно запустить из графического интерфейса, дважды щелкнув значок \verb|rars.jar|, представляющий этот исполняемый файл JAR. IDE предоставляет интуитивно понятные базовые возможности редактирования, сборки и выполнения. Ниже представлены некоторые доступные функции:
\begin{itemize}
    \item \textbf{Меню и панель инструментов.} Большинство элементов меню имеют одинаковые значки на панели инструментов. Если функция значка на панели инструментов не очевидна, просто наведите на него указатель мыши, и вскоре появится всплывающая подсказка. Почти все пункты меню также имеют сочетания клавиш. Любой пункт меню, не подходящий в данной ситуации, отключается.
    \item \textbf{Редактор.} RARS включает два встроенных текстовых редактора. Редактор по умолчанию имеет цветовую подсветку большинства языковых элементов с учетом синтаксиса и всплывающие инструкции. Помимо этого имеется ранее разработанный универсальный текстовый редактор без этих функций. Его можно выбрать в диалоговом окне <<Настройки редактора>>. Он поддерживает один шрифт, который можно изменить в диалоговом окне настроек редактора. Нижняя граница каждого из этих редакторов включает строку курсора и положение столбца. Также отображаются номера строк. \textit{Можно использовать внешний редактор. RARS предоставляет удобную настройку, которая будет автоматически собирать файл, как только он будет открыт. См. меню <<Настройки>>.}
    \item \textbf{Области сообщений.} В нижней части экрана есть две области сообщений с вкладками. Вкладка <<\textbf{Run~I/O}>> используется во время выполнения для отображения ввода--вывода консоли по мере выполнения программы. Также имеется возможность осуществлять консольный ввод во всплывающем диалоговом окне, а затем отображать его в области сообщений. Вкладка с названием <<\textbf{Сообщения~RARS}>> используется для других сообщений, таких как ошибки сборки или выполнения, а также для информационных сообщений. Вы можете нажать на сообщения об ошибках сборки, чтобы выбрать соответствующую строку кода в редакторе.

    \item \textbf{Регистры.} Регистры отображаются постоянно, даже тогда, когда программа редактируется, а не запускается. При написании программы это служит полезным справочником по именам регистров и их обычному использованию (можно навести указатель мыши на имя регистра, чтобы увидеть всплывающие подсказки). Имеется три вкладки регистров:
    \begin{itemize}
        \item файл регистров, содержащий целочисленные регистры от \verb|x0| до \verb|x31|, регистры \verb|LO| и \verb|HI|, а также счетчик программ (\verb|pc|);
        \item \textit{выбранные регистры сопроцессора 0 (исключения и прерывания)};
        \item регистры с плавающей запятой сопроцессора 1.
    \end{itemize}
    \item \textbf{Сборка.} Выберите <<Собрать>> в меню <<Выполнить>> или соответствующий значок на панели инструментов, чтобы собрать файл, который в данный момент находится на вкладке <<Правка>>. Файлы в текущем каталоге. Если параметр <<Собрать все>> включен, ассемблер соберет текущий файл как <<основную>> программу, а также соберет все другие файлы сборки (\verb|*.asm|; \verb|*.s|) в том же каталоге. А с опцией <<Assemble Open>>, открытый в данный момент ассемблер также соберет открытые в данный момент файлы. Результаты связаны, и если все эти операции были успешными, программа может быть выполнена. На метки, объявленные глобальными с помощью директивы \verb|.globl|, можно ссылаться в любом другом файле проекта. Существует также параметр, разрешающий автоматическую загрузку и сборку выбранного файла обработчика исключений.
    \item \textbf{Выполнение.} После успешной сборки программы инициализируются регистры и заполняются три окна на вкладке <<Выполнение>>: текстовый сегмент, сегмент данных и метки программы. Основные функции времени выполнения описаны ниже.
    \item \textbf{Окно меток.} Отображение окна меток (таблица символов) управляется через меню настроек. При отображении вы можете щелкнуть любую метку или связанный с ней адрес, чтобы отцентрировать и выделить содержимое этого адреса в окне «Текстовый сегмент» или в окне «Сегмент данных» в зависимости от ситуации.
\end{itemize}
    Ассемблер и симулятор вызываются из IDE, когда вы выбираете операции Assemble, Go или Step в меню Run или соответствующие им значки на панели инструментов или сочетания клавиш. Сообщения RARS отображаются на вкладке Сообщения RARS области сообщений в нижней части экрана. Ввод и вывод консоли среды выполнения обрабатывается на вкладке Run I/O.

\section{Интерактивные функции отладки}

RARS предоставляет множество функций для интерактивной отладки на панели <<Выполнение>>:
\begin{itemize}
    \item В пошаговом режиме выделяется следующая команда для выполнения, а содержимое памяти обновляется на каждом шаге.
    \item Для непрерывного выполнения выбирает опция <<Go>>. Еe также можно использовать для продолжения выполения из состояния паузы (шаг, точка останова, пауза).
    \item Точки останова легко устанавливаются и сбрасываются с помощью флажков рядом с каждой инструкцией, отображаемой в окне <<Текстовый сегмент>>. Можно временно приостановить точки останова, используя опцию <<Toggle Breakpoints>> в меню <<Выполнить>> или щелкнув заголовок столбца <<Bkpt>> в окне <<Текстовый сегмент>>. Точки останова можно снова активировать повторным выбором.
    \item При работе в режиме <<Go>> вы можете выбрать скорость симуляции с помощью ползунка <<Run Speed>>. Доступные скорости варьируются от 0,05 инструкций в секунду (20 секунд между шагами) до 30 инструкций в секунду, а выше этого предлагается <<неограниченная>> скорость. При использовании <<неограниченной>> скорости подсветка кода и обновление отображения памяти отключаются во время моделирования (но выполняется очень быстро!). При достижении точки останова происходит выделение и обновление. Скорость работы можно регулировать во время работы программы.
    \item При работе в режиме <<Go>> вы можете в любой момент приостановить или остановить симуляцию, используя функции <<Pause>> или <<Stop>>. Первый приостановит выполнение и обновит отображение, как если бы вы выполняли пошаговое выполнение или достигли точки останова. Последний завершит выполнение и отобразит окончательные значения памяти и регистров. При работе на «неограниченной» скорости система может не реагировать немедленно, но ответит.
    \item У вас есть возможность интерактивно шагать <<назад>> через выполнение программы по одной инструкции за раз, чтобы <<отменить>> шаги выполнения. Он будет буферизовать до 2000 самых последних шагов выполнения (это ограничение хранится в файле свойств и может быть изменено). Он отменит изменения, внесенные в память, регистры или CSR (однако в настоящее время состояние прерывания не сохраняется), но не консольный или файловый ввод-вывод. Это должно быть отличным средством отладки. Он доступен в любое время приостановки выполнения и при завершении (даже если оно было прекращено из-за исключения).
    \item Когда выполнение программы приостановлено или остановлено, выберите <<Сброс>>, чтобы сбросить все ячейки памяти и регистры до их исходных значений после сборки. Фактически <<Reset>> реализуется пересборкой программы.
    \item Адреса и значения памяти, а также значения регистров можно просматривать как в десятичном, так и в шестнадцатеричном формате. Все данные хранятся в порядке байтов с прямым порядком байтов (каждое слово состоит из байта 3, за которым следует байт 2, затем 1, затем 0). Обратите внимание, что каждое слово может содержать 4 символа строки, и эти 4 символа будут отображаться в порядке, обратном порядку строкового литерала.
    \item Содержимое сегмента данных отображается по 512 байт за раз (с прокруткой), начиная с базового адреса сегмента данных (0x10010000). Кнопки навигации предназначены для перехода к следующему разделу памяти, предыдущему или возврату к исходному (домашнему) диапазону. Поле со списком также предоставляется для просмотра содержимого памяти рядом с указателем стека (содержимое регистра sp), глобальным указателем (содержимое регистра gp), базовым адресом кучи (0x10040000), глобальными переменными .extern (0x10000000) или отображением на память устройств ввода--выводв (MMIO, 0xFFFF0000). Содержимое необработанного текстового сегмента также может быть отображено.
    \item Содержимое любого слова памяти сегмента данных и почти любого регистра можно изменить, отредактировав отображаемую ячейку таблицы. Дважды щелкните ячейку, чтобы отредактировать ее, и нажмите клавишу Enter, когда закончите вводить новое значение. Если вы введете недопустимое 32-разрядное целое число, в ячейке появится слово INVALID, и содержимое памяти/регистра не изменится. Значения можно вводить как в десятичном, так и в шестнадцатеричном формате (начальный "0x"). Отрицательные шестнадцатеричные значения можно вводить либо в формате дополнения до двух, либо в формате со знаком. Обратите внимание, что три целочисленных регистра (ноль, программный счетчик, адрес возврата) не могут быть отредактированы.
    \item Если включен параметр <<Самоизменяющийся код>> (отключен по умолчанию, смотрите в меню <<Настройки>>), двоичный код текстового сегмента можно изменить, используя тот же метод, описанный выше. Его также можно изменить, дважды щелкнув ячейку в столбце <<Код>> дисплея <<Текстовый сегмент>>.
    \item Содержимое ячеек, представляющих регистры с плавающей запятой, можно редактировать, как описано выше, и оно будет принимать допустимые шестнадцатеричные или десятичные значения с плавающей запятой. Поскольку каждый регистр двойной точности перекрывает два регистра одинарной точности, любые изменения в регистре двойной точности повлияют на одно или оба отображаемых содержимого соответствующих регистров одинарной точности. Изменения в регистре одинарной точности повлияют на отображение соответствующего ему регистра двойной точности. Значения, введенные в шестнадцатеричном формате, должны соответствовать формату IEEE-754. Значения, введенные в десятичном формате, вводятся с использованием десятичной точки и E-нотации (например, 12,5e3 — это 12,5 умножить на 10 в кубе).
    \item Содержимое ячейки можно редактировать во время выполнения программы, и после принятия оно будет применяться, начиная со следующей выполняемой инструкции.
    \item Нажатие на элемент окна <<Ярлыки>> приведет к тому, что местоположение, связанное с этим ярлыком, будет центрировано и выделено в окне <<Текстовый сегмент>> или <<Сегмент данных>> в зависимости от ситуации. Обратите внимание, что окно <<Ярлыки>> не отображается по умолчанию, но его можно открыть, выбрав его в меню <<Настройки>>.
\end{itemize}

\section{Дополнительные возможности: подключаемые инструменты}

RARS может запускать стороннее программное обеспечение, которое взаимодействует с исполняемой программой и системными ресурсами. Требования к такой программе:
\begin{enumerate}
    \item Он реализует интерфейс rars.tools.Tool.
    \item Это часть пакета rars.tools.
    \item Он аккуратно компилируется в файл «.class», хранящийся в каталоге rars/tools.
\end{enumerate}
RARS обнаружит все подходящие инструменты при запуске и включит их в свое меню «Инструменты». Когда выбран пункт меню инструмента, его экземпляр будет создан с использованием его конструктора без аргументов, и будет вызван его метод action(). Если при запуске не найдено подходящих инструментов, меню «Инструменты» не появится.

Чтобы использовать такой инструмент, загрузите и соберите интересующую вас программу, затем выберите нужный инструмент в меню «Инструменты». Откроется окно инструмента, и в зависимости от того, как он написан, его либо нужно будет «подключить» к программе, нажав кнопку, либо он уже будет подключен. Запустите программу, как обычно, чтобы инициировать взаимодействие инструмента с исполняемой программой.

Абстрактный класс rars.tools.AbstractToolAndApplication включен в дистрибутив RARS, чтобы обеспечить существенную основу для реализации вашего собственного Инструмента. Подкласс, который расширяет его, реализуя как минимум два его абстрактных метода, может быть запущен не только из меню «Инструменты», но и как отдельное приложение, использующее ассемблер и симулятор RARS в фоновом режиме.

Несколько инструментов, разработанных на основе подкласса AbstractMarsToolAndApplication, включены в RARS: введение в инструменты, симулятор кэша данных, визуализатор ссылок на память и инструмент с плавающей запятой. Последний весьма полезен, даже если он не подключен к программе, потому что он отображает двоичные, шестнадцатеричные и десятичные представления для 32-битного значения с плавающей запятой; когда любой из них изменяется, два других также обновляются.

\subsection{Добавление дополнительных системных вызовов}

Системные вызовы (инструкция ecall) реализованы с использованием метода, аналогичного инструментальному. Это позволяет любому добавить новый системный вызов, определив новый класс, отвечающий следующим требованиям:
\begin{enumerate}
    \item Он расширяет класс rars.riscv.AbstractSyscall.
    \item Это часть пакета rars.riscv.syscalls.
    \item Он аккуратно компилируется в файл ".class", хранящийся в каталоге rars/riscv/syscalls.
    \item Запись добавлена в Syscall.properties.
\end{enumerate}

\subsection{Расширение набора инструкций}

Вы можете добавить настраиваемые псевдоинструкции в набор инструкций, отредактировав файл \verb|PseudoOps.txt|. Форматы спецификаций инструкций объясняются в самом файле. Спецификация псевдоинструкции занимает одну строку. Он состоит из примера инструкции, созданной с использованием доступных символов спецификации инструкции, за которым следует список основных инструкций, разделенных точкой с запятой, до которых она будет расширена. Каждый из них представляет собой шаблон инструкции, построенный с использованием символов спецификации инструкции в сочетании со специальными символами спецификации шаблона. Последние допускают подстановку во время сборки программы операндов из пользовательской программы в расширенную псевдоинструкцию.

PseudoOps.txt считывается и обрабатывается при запуске, и если спецификация имеет неправильный формат, будут выдаваться сообщения об ошибках. Обратите внимание, что если вы хотите отредактировать его, вам сначала нужно извлечь его из файла JAR.

\section{Использование RARS из командной строки}

RARS можно запускать из интерпретатора командной строки для сборки и выполнения программы в пакетном режиме. Формат запуска:
\begin{verbatim}
    java -jar rars.jar [options] program.asm [more files...] \
                       [ pa arg1 [more args...]]
\end{verbatim}


Элементы в \verb|[ ]| являются необязательными. Допустимые параметры (без учета регистра, разделенные пробелами):

Таблица~\ref{table-option} описывает используемые опции.

\begin{table}[h]
    \caption{Опции командной строки запуска RARS}
    \centering
    \begin{tabularx}{\textwidth}{|c|X|}
        \hline
        \textbf{Опция} & \textbf{Описание} \\
        \hline \hline
        \texttt{а} & Только собрать, не выполнять \\
        \hline
        \texttt{ae<n>} & Завершает RARS целочисленным кодом выхода <n>, если возникает ошибка сборки \\
        \hline
        \texttt{ascii} & Отображать содержимое памяти или регистров, интерпретируемое как коды ASCII \\
        \hline
        \texttt{b} & brief --- не отображать адрес регистра/памяти вместе с содержимым \\
        \hline
        \texttt{d} & отображать операторы отладки RARS \\
        \hline
        \texttt{dec} & отображать содержимое памяти или регистра в десятичном формате \\
        \hline
        \texttt{dump} &  <сегмент> <формат> <файл> --- дамп памяти указанного сегмента памяти в указанном формате в указанный файл. Вариант может повторяться.
        Дамп происходит в конце симуляции, если не используется опция «a».
        Сегмент и формат чувствительны к регистру. Возможные значения:
        <сегмент> = .text, .data или диапазон, например 0x400000-0x10000000
        <format> = SegmentWindow, HexText, AsciiText, HEX, Binary, BinaryText \\
        \hline
        \texttt{g} & включить режим графического интерфейса \\
        \hline
        \texttt{h} & показать эту справку. Использовать отдельно без имени файла. \\
        \hline
        \texttt{hex} & отображать содержимое памяти или регистра в шестнадцатеричном формате (по умолчанию) \\
        \hline
        \texttt{ic} & отображать количество основных инструкций, "исполненных" \\
        \hline
        \texttt{mc <config>} & установить конфигурацию памяти. Аргумент <config> равен чувствительны к регистру и возможные значения: Default для значения по умолчанию 32-битное адресное пространство, CompactDataAtZero для памяти 32 КБ с сегмент данных по адресу 0 или CompactTextAtZero для 32 КБ
        памяти с текстовым сегментом по адресу 0. \\
        \hline
        \texttt{me} & отображать сообщения RARS в стандартном формате err вместо стандартного вывода. Может отделять сообщения от вывода программы с помощью перенаправления \\
        \hline
        \texttt{nc} & не отображать уведомление об авторских правах (для более чистого перенаправленного/конвейерного вывода). \\
        \hline
        \texttt{np} & использование псевдоинструкций и форматов не разрешено \\
        \hline
        \texttt{p} & Режим проекта --- собрать все файлы в том же каталоге, что и данный файл. \\
        \hline
        \texttt{se<n>} & завершить RARS целочисленным кодом выхода <n>, если возникает ошибка симуляции (запуска). \\
        \hline
        \texttt{sm} & начать выполнение с оператора с глобальной меткой main, если она определена \\
        \hline
        \texttt{smc} & самомодифицирующийся код --- программа может записывать и переходить как к тексту, так и к сегменту данных. \\
        \hline
    \end{tabularx}
\label{table-option}
\end{table}

\begin{table}[h]
    \caption{Опции командной строки запуска RARS (продолжение)}
    \centering
    \begin{tabularx}{\textwidth}{|c|X|}
        \hline
        \textbf{Опция} & \textbf{Описание} \\
        \hline \hline
        \texttt{rv64} & включает 64-битную сборку и исполняемые файлы (не полностью совместим с rv32). \\
        \hline
        \texttt{<n>} & где <n> целочисленное максимальное количество шагов для моделирования. Если 0, отрицательное или не указано, максимума нет. \\
        \hline
        \texttt{x<reg>} & где <reg> номер или имя (например, 5, t3, f10) регистра, содержимое которого будет отображаться в конце выполнения. Вариант может повторяться. \\
        \hline
        \verb|<reg_name>| & где \verb|<reg_name>| - это имя (например, t3, f10) регистра, содержимое которого будет отображаться в конце выполнения. Вариант может повторяться. \\
        \hline
        \texttt{<m>-<n>} & диапазон адресов памяти от <m> до <n>, содержимое которого будет отображаться в конце выполнения. <m> и <n> могут быть шестнадцатеричными или десятичными, должны быть на границе слова, <m> <= <n>. Вариант может повторяться. \\
        \hline
        \texttt{pa} & Аргументы программы следуют в списке, разделенном пробелами. Эта опция должна быть помещена ПОСЛЕ ВСЕХ ИМЕН ФАЙЛОВ, потому что все, что следует за ней, интерпретируется как программный аргумент, который должен быть доступен программе во время выполнения. \\
        \hline
    \end{tabularx}
    \label{table-option2}
\end{table}

Если указано более одного имени файла, первое считается основным, если только в одном из файлов не определена метка глобального оператора main. Обработчик исключений не собирается автоматически. Добавьте его в список файлов. Используемые здесь параметры не влияют на значения меню настроек RARS и наоборот.

\section{Ограничения ассемблера и эмулятора RARS}
RARS разработан для эмуляции версии RV32IMFN. Ограничения RARS версии 1.0 включают:
\begin{itemize}
    \item Сегменты памяти (текст, данные, стек) ограничены 4 МБ каждый, начиная с соответствующих базовых адресов.
    \item Конвейерного режима нет.
    \item Если вы откроете файл, который является ссылкой или ярлыком на другой файл, RARS не откроет целевой файл. Диалоговое окно открытия файла реализовано с использованием Java Swing JFileChooser, который не поддерживает ссылки.
    \item Очень немногие изменения конфигурации, кроме тех, что в меню настроек, сохраняются от одного сеанса к другому. Настройки редактора, включая настройки шрифта и отображение номеров строк, сохраняются.
    \item IDE будет работать только с ассемблером RARS. Его нельзя использовать ни с каким другим компилятором, ассемблером или симулятором. Ассемблер и симулятор RARS можно использовать либо через IDE, либо из командной строки.
    \item Поддержка операций с плавающей запятой не полностью совместима, поскольку Java не обеспечивает достаточно низкоуровневый доступ к операциям с плавающей запятой.
    \item Регистры управления и состояния не могут быть указаны в инструкциях по имени. Это полностью решаемая проблема.
    \item Поддержка прерываний имеет некоторые недостатки, необходимо проделать большую работу, чтобы приблизить ее к спецификации.
    \item Ошибка: Подсветка сообщений об ошибках не выбирает автоматически код первой ошибки сборки, если файл, содержащий ошибку, не открыт во время сборки (сборка-при-открытии, сборка-все).
    \item Ошибка: в редакторе произошла утечка памяти. Несколько разных людей независимо друг от друга сообщали об одном и том же: сильное замедление реакции редактора во время продолжительного интерактивного сеанса. Если выйти из RARS и перезапустить его, это поведение исчезает, и редактор мгновенно реагирует на действия.
\end{itemize}
