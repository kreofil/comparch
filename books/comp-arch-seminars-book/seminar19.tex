\chapter[19]{Семинар 19. Микроархитектура. Предсказание переходов. Кеширование}

Целью семинара является продолжение изучения особенностей организации и использования интерфейса передачи сообщений на основе библиотеки Message Passing Inteface (MPI).

Изучение примеров с MPI, прилагающихся к лекциям. Предыдущие и эти примеры достаточно короткие. Поэтому можно просто пробежаться. Вполне возможно, что MPI придется установить. Пока на Убунте у меня не удалось настроить OpenMPI, хотя сам пакет установился проблемы оказались с настройкой путей к заголовочным файлам. Пока же не проходит компиляция. Не находит этих путей. Конфигурировать пробовал, но пока результа не достигнут. Буду пробовать еще. Не получится --- попробую установить и настроить MPICH. Это еще одна версия MPI, доступная в Убунте. Когда-то она была основной в Linux. Результаты сообщу. Текст подправлю. Также хочу попробовать под Virtual Box Simply Linux.

Есть еще один дежурный вариант, который может устроить всех: установка MPI на рабочей системе. Он есть как под Виндой, так и под Яблоком. Это в принципе можно учесть при раздаче домашнего задания. Если у тебя получится под Линуксом --- попробуй Винду. Можно потом прописать в семинарах. Поищи ссылки в сети. Можешь их вставить в свою презентацию и выложить в семинар. Пусть делают ДЗ на своих ОС. Это непринципиально.

Кстати, отсутствующие а ЛМС семинары я постараюсь обозначить по темам и можно туда включить твои презентации. Лекционные примеры, наверное, переносить не буду, но в последующем что-то можно поискать и добавить. В частности, домашние задания.

\textbf{\textit{Если не получится ничего установить в классе, то можно, как я планирую, посмотреть функционирование кода на проекторе со своего компьютера.}}

Примерный перечень вопросов (часть перенесена из предыдущего семинара, так как не рассматривалась на лекции):
\begin{enumerate}
    \item Пример 01 (01. Простое распределенное приложение: "Привет от MPI"). На нем име можно поиграться как с параметрами файла хостов, так и числом запускаемых процессов. Показать, что при ограниченном числе слотов в конфигурационном файле запуск большего числа процессов не проходит. То же можно просмотреть и без указания слотов в команде, когда раределение процессов по узлам идет по кругу, но до предела, определяемого умолчанием. Расширить этот предел до <<неограниченного>> числа можно использованием опции \verb|--oversubscribe|. Здесь удобно тем, что при превышении числа слотов, указанного в конфигурационном файле, выдается сообщение, подсказывающее о вариантах использования. Еще можно попытаться задать огромное число процессов. Например, 1000. Система должна ругнуться на ограниченность процессов и пайпов, которые она может запустить. Особенно на пайпы, которые, видимо используются внутри MPI.
    \item Пример 02 (02. MPI. Пример передачи и приема сообщений). Пример ориентирован на взаимодействие только двух процессов по простейшему блокирующему передаче и приему. В данном случае интересно обсудить, каким образом осуществляется ограничение процессов. Можно также попробовать предварительно до приема прочитать статус. Посмотрет (count) переданное число байт. Но в целом это базовая схема для различных других вариантов использования. Как варианты, можно предложить сделать модификации, в которых вместо строки осуществляется передача целых, действительных чисел.
    \item Пример 03 (03. MPI. Использование барьеров). Барьерная задача очень проста. Но в принципе на ней можно увидеть, что будет, если барьер убрать. Тогда последнее сообщение, стоящее после барьера, будет выведено не в конце.
    \item Пример 04 (04. MPI. Широковещательная рассылка). Детально разобрать пример, программы, осуществляющей широковещательную рассылку. Можно раскомментировать выводы, демонстрирующие количество процессов коммуникаторах. Позапускать программу с разным числом процессов (начиная с 1). Четным и нечетным. Оценить поведение.
    \item Пример 05 (05. MPI. Вычисление суммы квадратов). Как вариант из предыдущих семинаров: можно увеличить нагрузк на цикл, добавив, например, вычисление квадратного корня или чего-то еще. Также интересно посмотреть вычисление на разном числе узлов с разным и сопоставить время вычислений при одинаковом числе элементов в массиве.
    \item Пример 06 (06. MPI. Вычисление числа Пи). Пример завершает работу при вводе нуля. На нем можно посмотреть широковещательную рассылку, использование суммирования и свертки с применением MPI. Также интересно несколько раз задать разную точность вычисления путем задания количества интервалов. Сопоставить время получения результата.
\end{enumerate}

\section{Домашнее задание}

Установить на домашней системе MPI и прислать отчет (в виде сканов), демонстрирующих выполнение программы осуществляющей пересылки текстовых сообщений между тремя потоками в соответствии со схемой, представленной на слайде 19 лекции по MPI. За основу можно взять пример 02.

