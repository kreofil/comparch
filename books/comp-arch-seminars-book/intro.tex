% Введение

\chapter* {Введение}
\addcontentsline{toc}{chapter}{Введение}

\section*{Назначение документа}

Приводится содержание семинарских занятий, проводимых по дисциплине "Архитектура вычислительных систем в 2023-2024 учебном году. Он является основой для проведения занятий семинаристами. Для каждого семинара представлены:
\begin{itemize}
    \item тема семинарского занятия;
    \item цель занятия;
    \item общий план занятия;
    \item содержание занятия (решаемые задачи, рассматриваемые вопросы) в соответствии с представленным планом, включающее последовательность действий, а также дополнительные комментарии и примечания к процессу проведения;
    \item описание домашнего задания;
    \item пример (или примеры) выполнения домашнего задания (зачастую это один из возможных вариантов);
    \item список основной и дополнительной литературы, рекомендуемой по теме семинара и необходимой для выполнения задания;
    \item список используемого аппаратного и программного обеспечения.
\end{itemize}

Данный документ не является догмой и жестким предписанием. Он определяет общий план и включает ряд пояснений, сформированных на основе предшествующего опыта. Поэтому он может уточняться и изменяться всеми участниками учебного процесса. Основной его целью является помощь в более быстром вхождении в процесс проведения семинаров тем кто впервые приступает к занятиям по данной дисциплине.

\textbf{Изменять, расширять, дополнять не только можно, но и нужно.} Задача --- сформировать общие алгоритмы, которые в дальнейшем можно использовать как паттерны, адаптируя их к текущим ситуациям. Изменение эти алгоритмов возможно еще и потому, что каждый год, опираясь на предыдущий опыт, идет изменение учебного процесса. Этот процесс изменения является постоянным, так как нельзя дважды войти в одну и ту же реку...

\section*{Краткое содержание семинаров}
\debate[Примечание]{По ходу написания этот список будет уточняться и корректироваться.}

В соответствии с учебным планом предполагается проведение пятнадцати семинаров. Предполагается, что это будут следующие темы:
\begin{enumerate}
	\item Эмулятор RARS. Первоначальное знакомство.
    \item Организация памяти. Методы адресации. Ветвления и переходы. Системные вызовы.
    \item Целочисленная арифметика. Одномерные и многомерные массивы. Простые алгоритмы.
    \item Подпрограммы. Стек. Кадр стека. Параметры. Локальные переменные.
    \item Немного об ассемблере. Многофайловые программы. Директивы. Макросы.
    \item Математический сопроцессор. Арифметика с плавающей точкой.
    \item Строки символов. Обработка символьных данных. Файлы.
    \item Обработка исключений.
    \item Ввод--вывод данных. Поллинг. Программирование ввода--вывода.
    \item Таймер. Прерывания по таймеру.
    \item Программирование ввода--вывода. Использование прерываний.
    \item Микроархитектура. Предсказание переходов. Кеширование.
    \item Микроархитектура. Конвейеризация.
    \item Поддержка многозадачности. Виртуализация. Многоядерность.
    \item Мультипроцессоры. Специализированные процессоры.
\end{enumerate}

\debate[Примечание]{Если во введении необходимы дополнительные краткие пояснения по темам, то можно добавить.}


%\subsection*{Процессор RISC-V и его отображение в эмуляторе RARS}
