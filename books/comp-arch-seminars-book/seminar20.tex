\chapter[20]{Семинар 20. Микроархитектура. Конвейеризация}

Целью семинара является сопоставление методов псевдопараллельного и параллельного програмирования на примере событийного подхода, используемого в различных системах (C\#, Qt и~др.) и многопоточного программировния на основе \texttt{pthread}. То есть, для упрощения вычислений и привязки к заданию свести тему к многопочности, а не параллелизму в общем виде.

Честно говоря, у меня начинается потеря связи между лекционным материалом и его отображением на практике, что связано с невозможностью воспроизвести на текущем оборудовании ряда задач, которые еще могу рассмотреть на лекции. Поэтому я решил попробовать посвятить семинар материалу, который можно использовать при выполнении четвертого задания, с одной стороны. С другой стороны этот материал можно также использовать для формирования представления о методах программирования, используемых в современных последовательных системах для организации псевдопараллельных вычислений.

Для этого я предлагаю рассмотреть библиотеку Qt и тот подход, который там используется для взаимодействия отдельных функций в асинхронном режиме. На примере программы имитации роста растений и поедающих эту растительность животных рассмотреть поведение системы и попытаться отобразить его (пока концептуально) в поведение многопоточной системы, аналогичной тем программам, которые реализуются в четвертом задании.

Оставшееся время (или, если что-то не пойдет с этим вариантом) можно посвятить разбору третьего задания. Я думаю, что сильно в это погружаться не стоит, так как ключевым является переход к концептуальному обсуждению схемы задачи, описывающей параллелизм и методов ее реализации в реальной многопоточной среде. То есть, рассмотреть на уровне общей структуры, не забираясь в само кодирование. Чтобы при выполнении своих заданий были некоторые стереотипы, куда идти.

Примерный перечень вопросов:
\begin{enumerate}
    \item Краткий рассказ об особенностях событийного программирования на примере практически любых систем. По имеющейся информации нынешние второкурсники изучали основы обработки событий в C\#. Проблемы могут возникнуть с теми, кто пришел с потока ИСП РАН. Можно кратко охарактеризовать на пальцах. Ты ведь знаешь C\# и другие. Можно сориентироваться и на библиотеки Qt. Есть у меня соответствующая лекция для первокурсников. Она будет в 13-м семинаре, как и соответствующие примеры. Обсудить, что программа по сути формируется как набор псевдопараллельных компонент, связанных между собой <<проводами>>, по которым передаются сообщения, являющиеся по сути сигналами. Реализация этого механизма осуществляется за счет формирования очередей сообщений, из которой некоторым диспетчером осуществляется последовательная выборка с запуском очередной функции или метода. При этом ОС система дополнительно управляет переключением потоков, отвечающих за различные другие функции, что в данном случае является несущественным для рассматриваемой программы, но поддерживает общий параллелизм. Псевдопараллелизм поддерживается тем, что сами выполняемые функции не являются нагруженными задачей. \textbf{Но сильно погружаться (заморачиваться) не нужно. На уровне интуиции.} \textit{Можно во время демонстрации примеров.}
    \item Далее (или сразу же) можно привести примеры игры elife, имитирующей поедание растущей травы. Есть (выложена) чужая реализация на JS и моя на Qt. Положил проекты на qmake и cmake. Исходя из этого можно обсудить внутреннюю структуру, например, обработки на JS. Запустить по ходу можно несколько раз. Можно одновременно оба на одном экране.
    \item После этого можно приступить в вариантам возможной реализации с примененим многопоточного программирования. Фиксируя, например, на доске, предлагаемые варианты. Пусть студенты изначально сами предложат версии того, как можно написать параллельную (многопоточную) программу. На первом этапе можно не вдаваться в детали, а породить именно варианты возможного воплощения сценария в потоках. То есть, то что им нужно сделать (сформировать хотя бы один сценарий) по заданию 4. Возможные варианты:
        \begin{itemize}
            \item Один поток --- поведение всей еды, другой --- поведение поедателей. Синхронизация при доступе к поляне.
            \item Каждый поток определяет один элемент еды или поедателя. Поляна --- набор данных вокруг которого идет синхронизация. Здесь речь пойдет о создании потоками новых потоков и завершении тех потоков при соответствующих условиях.
            \item Каждая клетка поляны - поток, который содержит в виде данных одно из состояний (пусто, еда, поедатель). Здесь возникает синхронизации с соседними клетками.
            \item Возможны и другие варианты. Пусть в начале предлагают свои. После этого, если останутся, соответствующие из выше предложенных.
        \end{itemize}
    \item После этого можно обсудить специфику каждого из сценариев. То есть, как и какие порождать потоки. Как их синхронизировать. Где проще организовать. Какой вариант обладает бОльшим параллелизмом. Какие варианты синхронизации возникают. В принципе здесь могут появиться и подварианты. Можно попробовать устроить конкурс: кому какой вариант больше нравится. И разбить анализ по подгруппам. Что у какой получится...
    \item Доводить до реализации не стоит. Но если будет время, то детали каких-то вариантов можно проработать. То есть, именно порождение концептуальных решений.
\end{enumerate}

Остаток занятия или перерыв можно оставить под обсуждение результатов выполнения задания 3

\section{Домашнее задание}

Скорее всего выдавать не стоит, так как дедлайн по заданию 4. Пусть используют идеи в своем варианте.

