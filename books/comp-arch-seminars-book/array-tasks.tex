\chapter*{Приложение А Задание 1. Целочисленная арифметика и массивы}
\addcontentsline{toc}{chapter}{Приложение А Задание 1. Целочисленная арифметика и массивы}

Разработать программу, которая получает одномерный массив \textbf{A$_N$}, после чего формирует из элементов массива \textbf{A} новый массив \textbf{B} по правилам, указанным в варианте, и выводит его. Память под массивы может выделяться статически, на стеке, в области кучи по выбору разработчика. При решении задачи необходимо использовать подпрограммы для реализации ввода, вывода и формирования нового массива. Ввод-вывод элементов реализовать в окно выполнения эмулятора с использованием его системных вызовов.

\begin{enumerate}
    \item Сформировать массив \textbf{B} из положительных элементов массива \textbf{А}.
    \item Сформировать массив \textbf{B} только из тех элементов массива \textbf{А}, которые не совпадают с его первым и последним элементами.
    \item  Сформировать массив \textbf{B} из сумм соседних элементов \textbf{A} по следующим правилам:
           \textbf{ B$_0$ = A$_0$ + A$_1$, B$_1$ = A$_1$ + A$_2$, ... }
    \item Массив \textbf{B} формируется по следующим правилам:
    \begin{itemize}
        \item \textbf{B$_i$ = 1, если A$_i$ > 0},
        \item \textbf{B$_i$ = -1, если A$_i$ < 0},
        \item \textbf{B$_i$ = 0, если A$_i$ = 0}.
    \end{itemize}
    \item Сформировать массив \textbf{B}, состоящий из элементов массива \textbf{А}, значение которых не совпадает с дополнительно введённым числом \textbf{X}.
    \item Сформировать массив \textbf{B}, состоящи из элементов массива \textbf{A}, значения которых кратны введенному числу \textbf{X}.
    \item Сформировать массив \textbf{B} из индексов положительных элементов массива \textbf{A}.
    \item Сформировать массив \textbf{B} по следующим правилам:
    \begin{itemize}
        \item \textbf{если A$_i$ > 5, то увеличить элемент B$_i$ на 5,},
        \item \textbf{eсли A$_i$ < -5, то уменьшить B$_i$ на 5,},
        \item \textbf{остальные B$_i$ обнулить.}.
    \end{itemize}
    \item Сформировать массив \textbf{B} из нечётных элементов массива \textbf{A}.
    \item Сформировать массив \textbf{B} из отрицательных элементов массива \textbf{A}, расположенных обратном порядке.
    \item Сформировать массив \textbf{B} из элементов \textbf{A}, расположенных в обратном порядке, исключая первый положительный элемент.
    \item Сформировать массив \textbf{B} из элементов массива \textbf{A}, исключив первый
    положительный и последний отрицательный элементы.
    \item Сформировать массив \textbf{B} из элементов массива \textbf{A}, за исключением элементов, значения которых совпадают с минимальным элементом массива \textbf{A}.
    \item Сформировать массив \textbf{B} из элементов массива \textbf{A} заменой всех отрицательных значений на максимум из массива \textbf{A}.
    \item Сформировать массив \textbf{B} из элементов массива \textbf{A} заменой всех нулевых элементов значением минимального элемента.
    \item Сформировать массив \textbf{B} из элементов массива \textbf{A}, заменой на среднее арифметическое тех значений, которые больше среднего арифметического.
    \item Сформировать массив \textbf{B} из элементов массива \textbf{A}, расположенных после последнего положительного элемента.
    \item  Сформировать массив \textbf{B} из элементов массива \textbf{A} уменьшением всех элементов, расположенных до первого положительного, на \textbf{5}.
    \item Сформировать массив \textbf{B} из элементов массива \textbf{A} заменой нулевых элементов, предшествующих первому отрицательному, единицей.
    \item  Сформировать массив \textbf{B} из элементов массива \textbf{A} перестановкой местами минимального и первого элементов.
    \item Сформировать отсортированный по возрастанию массив \textbf{B} из элементов массива \textbf{A}.
    \item  Сформировать отсортированный по убыванию массив \textbf{B} из элементов массива~\textbf{A}.
    \item Сформировать массив \textbf{B}, элементы которого являются расстояниями пройденными телом при свободном падении на землю за время в секундах, указанное в массиве \textbf{A}. Решение получить в целых числах, приняв ускорение свободного падения за \textbf{10}.
    \item Сформировать массив \textbf{B} из элементов массива \textbf{A} поменяв местами элементы, стоящие на чётных и нечётных местах:
            \textbf{A$_0$ $\leftrightarrow$ A$_1$; A$_2$ $\leftrightarrow$ A$_3$ ...}
    \item Сформировать массив \textbf{B} из элементов массива \textbf{A} заменив все положительные числа значением \textbf{2}, а отрицательные — увеличить на \textbf{5}.
    \item Сформировать массив \textbf{B} из сумм трех соседних элементов массива \textbf{A}, сумма значений которых максимальна. Если элементов в массиве \textbf{А} менее трёх, то заполнить массив \textbf{В} нулями.
    \item Сформировать массив \textbf{B} из элементов массива \textbf{A}. Элементы массива \textbf{А}, оканчивающиеся цифрой \textbf{4}, уменьшить вдвое.
    \item Сформировать массив \textbf{B} из элементов массива \textbf{A}, которые образуют неубывающую последовательность. Неубывающей последовательностью считать элементы идущие подряд, которые равны между собой или каждый последующий больше предыдущего.
    \item Сформировать массив \textbf{B} из произведения соседних элементов \textbf{A} по следующему правилу: \textbf{B$_0$ = A$_0$ * A$_m$ , B$_1$ = A$_1$ * A$_m$, ...}, где \textbf{m} – либо номер первого четного отрицательного элемента массива А, либо номер последнего элемента, если в массиве А нет отрицательных элементов.
    \item Сформировать массив \textbf{B} из тех элементов массива \textbf{A}, которые больше, чем элементы, стоящие перед ними.
    \item Сформировать массив \textbf{B} из элементов массива \textbf{A} в следующем порядке: сначала заполняем массив \textbf{В} числами, стоящими на нечетных местах, а затем - стоящие на четных местах в массиве \textbf{А}.
    \item Сформировать массив \textbf{B} из элементов массива \textbf{A}, которые меньше суммы элементов, расположенных на четных местах.
    \item Сформировать массив \textbf{B} на основе элементов массива \textbf{A}, полученных как разность соседних элементов.
    \item Сформировать массив \textbf{B} из элементов массива \textbf{A} заменив элементы на четным местах суммой всех положительных элементов, а элементы на нечетных местах суммой отрицательных элементов.
    \item Сформировать массив \textbf{B} из элементов массива \textbf{A} сгруппировав положительные элементы массива \textbf{А} в начале массива \textbf{В}, а отрицательные --- в конце.
    \item Сформировать массив \textbf{B} из элементов массива \textbf{A} сгруппировав элементы с четными индексами в начале массива \textbf{В} , а элементы с нечетными индексами сгруппировать в конце массива \textbf{В}.
    \item Сформировать массив \textbf{B} из элементов массива \textbf{A} в следующем порядке: элементы с индексами \textbf{i $\le$ (N + 1)/2} переместить на позиции с четными индексами массива \textbf{В} с сохранением их исходного порядка относительно друг друга, а оставшиеся элементы (\textbf{i $>$ (N + 1)/2}) разместить на позициях с нечетными индексами массива \textbf{В} также с сохранением их исходного порядка.
    \item Сформировать массив \textbf{B} из элементов массива \textbf{A}, которые одновременно имеют четные и отрицательные значения.
    \item Сформировать массив \textbf{B}, элементы которого являются площадью прямоугольников со сторонами указанные в массиве \textbf{A}: \textbf{B$_i$ = A$_i$ * A$_{i+1}$}.
    \item Сформировать массив \textbf{B} из суммы соседних элементов \textbf{A} по следующему правилу: \textbf{B$_0$ = A$_0$, B$_1$ = A$_0$ + A$_1$, ..., B$_m$ = A$_0$ + ... + A$_m$}, где \textbf{m} – номер
    первого элемента массива \textbf{А} большего, чем среднее арифметического этого массива.
\end{enumerate}
