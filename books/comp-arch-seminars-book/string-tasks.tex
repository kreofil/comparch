\chapter*{Приложение В Задание 3. Обработка строк символов}
\addcontentsline{toc}{chapter}{Приложение В Задание 3. Обработка строк символов}

\textit{ASCII-строка --- строка, содержащая символы таблицы кодировки ASCII.} (\url{https://ru.wikipedia.org/wiki/ASCII}). Размер строки может быть достаточно большим, чтобы вмещать многостраничные тексты, например, главы из книг, если задача связана с использованием файлов или строк, порождаемых генератором случайных чисел. Тексты при этом могут не нести смыслового содержания. Для обработки в программе предлагается использовать данные, содержащие символы только из первой половины таблицы (коды в диапазоне 0--127$_{10}$), что связано с использованием кодировки UTF-8 в ОС Linux в качестве основной. Символы, содержащие коды выше 127$_{10}$, должны отсутствовать во входных данных кроме оговоренных специально случаев.

\begin{enumerate}
    \item Разработать программу, которая <<переворачивает>> заданную позициями \textbf{N$_1$--N$_2$} часть ASCII--строки символов (\textbf{N$_1$, N$_2$} вводятся как параметры).
    \item Разработать программу, находящую в заданной ASCII--строке первую  при обходе \textbf{от конца к началу} последовательность \textbf{N} символов, каждый элемент которой определяется по условию <<больше предшествующего>> (\textbf{N} вводится в качестве параметра).
    \item Разработать программу, находящую в заданной ASCII-строке первую слева направо последовательность \textbf{N} символов, каждый элемент которой определяется по условию <<меньше предшествующего>> (\textit{\textbf{N} вводится как отдельный параметр}).
    \item Разработать программу, находящую в заданной ASCII-строке последнюю при перемещении слева направо последовательность \textbf{N} символов, каждый элемент которой определяется по условию <<больше предшествующего>> (\textit{\textbf{N} вводится как отдельный параметр}).
    \item Разработать программу, заменяющую все строчные гласные буквы в заданной ASCII-строке заглавными.
    \item Разработать программу, заменяющую все согласные буквы в заданной ASCII-строке их \textbf{ASCII кодами в десятичной системе счисления}.
    \item Разработать программу, заменяющую все гласные буквы в заданной ASCII-строке их \textbf{ASCII кодами в шестнадцатиричной системе счисления}. Код каждого символа задавать в формате <<\textbf{0xDD}>>, где \textbf{D} --- шестнадцатиричная цифра от 0 до F.
    \item Разработать программу, заменяющую все цифры (за исключением нуля) в заданной ASCII-строке \textbf{римскими цифрами}. То есть, соответствующим комбинациями букв для цифр от 1 до 9.
    \item Разработать программу, которая <<\textbf{переворачивает на месте}>> заданную ASCII-строку символов (не копируя строку в другой буфер).
    \item Разработать программу, которая меняет на обратный порядок следования символов \textbf{каждого слова} в ASCII-строке символов. Порядок слов остается неизменным. Слова состоят только из букв. Разделителями слов являются все прочие символы.
    \item Разработать программу вычисления \textbf{отдельно количества гласных и согласных букв} в ASCII-строке.
    \item Разработать программу, определяющую \textbf{минимальный и максимальный (по числовому значению) символ в заданной} ASCII-строке.
    \item Разработать программу, заменяющую все строчные буквы в заданной ASCII-строке прописными, а прописные буквы --- строчными.
    \item Разработать программу, вычисляющую отдельно \textbf{число прописных и строчных букв} в заданной ASCII-строке.
    \item Разработать программу, которая на основе заданной ASCII-строки символов, решает вопрос, является ли данная строка \textbf{палиндромом}.
    \item Разработать программу, которая вычисляет \textbf{количество цифр и букв} в заданной ASCII-строке.
    \item Разработать программу, которая \textbf{меняет на обратный порядок следование слов} в ASCII-строке символов.
    \item Разработать программу, \textbf{заменяющую все согласные буквы в заданной ASCII-строке на заглавные}.
    \item Разработать программу, вычисляющую число вхождений различных \textbf{отображаемых символов} в заданной ASCII-строке.
    \item Разработать программу, вычисляющую \textbf{число вхождений различных цифр} в заданной ASCII-строке.
    \item Разработать программу, осуществляющую \textbf{нахождение суммы всех цифр} в заданной ASCII-строке.
    \item Разработать программу, вычисляющую \textbf{число вхождений различных знаков препинания} в заданной ASCII-строке.
    \item Разработать программу, которая ищет в ASCII-строке заданную подстроку и возвращает \textbf{индекс первого символа первого вхождения подстроки в строке}. Подстрока вводится как параметр.
    \item Разработать программу, которая ищет в ASCII-строке заданную подстроку и возвращает \textbf{список индексов первого символа для всех вхождений} подстроки в строке. Подстрока вводится как параметр.
    \item Разработать программу, которая определяет в ASCII-строке \textbf{частоту встречаемости различных идентификаторов}, являющихся словами, состоящими из букв и цифр, начинающихся с буквы. Разделителями являются все другие символы. Для тестирования можно использовать программы, написанные на различных языках программирования.
    \item Разработать программу, которая определяет \textbf{количество целых чисел} в ASCII-строке. числа состоят из цифр от 0 до 9. Разделителями являются все другие символы.
    \item Разработать программу, которая определяет \textbf{частоту встречаемости (сколько раз встретилось в тексте) пяти ключевых слов языка программирования C}, в произвольной  ASCII-строке. Ключевые слова не должны являться частью идентификаторов. Пять искомых ключевых слов выбрать \textbf{по своему усмотрению}. Тестировать можно на файлах программ.
    \item Разработать программу, которая в заданной ASCII-строке определяет \textbf{корректность вложенности круглых скобок} <<\verb|(|>> и <<\verb|)|>>. Необходимо учесть, что вложенные скобки могут образовывать в тексте различные группы. Например: \verb|...(...)...(...)...|.
    \item Разработать программу, которая ищет в ASCII-строке \textbf{слова --- палиндромы и формирует из них новую строку}, в которой слова разделяются пробелами. Слова состоят из букв. Все остальные символы являются разделителями слов.
    \item Разработать программу, которая ищет в ASCII-строке слова, \textbf{начинающиеся с заглавной буквы и формирует из них новую строку}, в которой слова разделяются пробелами. Слова состоят из букв. Все остальные символы являются разделителями слов.
    \item Разработать программу, которая ищет в ASCII-строке \textbf{целые числа и формирует из них новую строку}, в которой числа разделяются знаком <<\verb|+|>>. Слова состоят из букв. Все остальные символы являются разделителями слов.
    \item Разработать программу, которая на основе анализа двух входных ASCII-строк формирует на выходе две другие строки. В первой из выводимых строк содержатся символы, которых нет во второй исходной строке. Во второй выводимой строке содержатся символы, отсутствующие в первой входной строке (\textbf{разности символов}). Каждый символ в соответствующей выходной строке должен встречаться только один раз.
    \item Разработать программу, которая на основе анализа двух ASCII-строк формирует на выходе строку, содержащую символы, присутствующие в обеих строках (\textbf{пересечение символов}). Каждый символ в соответствующей выходной строке должен встречаться только один раз.
    \item Разработать программу, которая на основе анализа двух ASCII-строк формирует на выходе строку, содержащую символы, присутствующие в одной или другой (\textbf{объединение символов}). Каждый символ в соответствующей выходной строке должен встречаться только один раз
\end{enumerate}
