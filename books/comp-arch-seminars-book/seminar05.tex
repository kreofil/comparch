\chapter[Подпрограммы. Начало]{Семинар 05. Подпрограммы. Передача параметров}

Целью семинара является изучение использования подпрограмм на уровне систмы команд.
На занятии предполагается рассмотреть следующие темы:
\begin{enumerate}
    \item Особенности вызова подпрограмм в системе команд RISC-V и возврата из подпрограмм.
    \item Использование подпрограмм без параметров и с параметрами. Достоинства и недостатки. Примеры.
    \item Соглашения о передаче фактических параметров и возврате результатов в архитектуре RISC-V.
\end{enumerate}

\section{Особенности вызова подпрограмм и возврата из подпрограмм}
\debate[Примечание]{Данную тему в первой итерации предлагается заимствовать из Г.~Курячего, что в общем-то уже и сделано. В дальнейшем, думаю, стоит уточнить примеры. Помимо этого вставить дополнительные более сложные примеры, раскрывающие особенности различных моментов использования подпрограмм. Все необходимые материалы уже сгруппированы в соответствующей презентации.}

В целом изложение выглядит следующим образом:
\begin{itemize}
    \item определение подпрограммы;
    \item отличие ассемблерных подпрограммы от подпрограмм, процедур, функций языков высокого уровня;
    \item описание подпрограмм в ассемблере процессора RISC-V (эмулятора RARS);
    \item команды и псевдокоманды вызова подпрограмм;
    \item команды и псевдокоманды возврата из подпрограмм
\end{itemize}

\section{Использование подпрограмм и без параметров. Достоинства и недостатки. Примеры}

Основной акцент сделать на примерах (изложены в презентациях). Показать на эмуляторе в пошаговом режиме, что происходит:
\begin{itemize}
    \item пример использования вызовов \verb|jal|  и \verb|jalr|;
    \item пример с отрезками, составляющими стороны треугольника (на нем можно показать изощренность возврата, что не всегда удобно в практическом решении из-за появления избыточных зависимостей);
\end{itemize}

\section{Соглашения о передаче фактических параметров и возврате результатов}

Рассказать об основных соглашениях, моделирующих передачу параметров внутрь подпрограмм, аналогичную использованию параметров в процедурах и функциях. Рассказать о соглашении использования регистров \verb|a*| как для передачи параметров, так и возврата результатов (\verb|a0|, \verb|a1|). Лишний раз напомнить, для чего используются соглашения.

\section{Выдача задания №~1}

Судя по дыре в расписании учебных занятий, минисессия будет проходить с 30 октября. Думаю, что на выполнение первого задания можно выделить 3 полных недели, поставив дедлайн 15 октября 23:59. График заданий сформирован и выложен в ЛМС. Также уже выложены варианты и требования к выполнению заданий. Все это ляжет и в репозиторий. Генератор вариантов тоже лежит. В принципе всю информацию по вариантам можно выложить в чаты и зафиксировать в текущих ведомостях. Основное --- объявить дефакто, что задание роздано. Я об этом также объявлю на лекции.

Для групп, которые в пятницу, распределить задания можно раньше. Хоть в понедельник, разослав им список номеров по группам. По почте каждой группе отдельно предпочтительнее. Я наверное тоже своей группе дополнительно вышлю, так как тогда будет документ, против которого возразить будет сложно.

\section{Домашнее задание}

Домашнее задание завязано на второй семинар по данной теме, который по сути является продолжением. Тема разбита и немного увеличена по содержанию из-за добавления семинарских занятий в этом году.
