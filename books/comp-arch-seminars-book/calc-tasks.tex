\chapter*{Приложение Б Задание 2. Вычисления с плавающей точкой}
\addcontentsline{toc}{chapter}{Приложение Б Задание 2. Вычисления с плавающей точкой}

Разработать программы на языках программирования C и Ассемблер, выполняющие вычисления над числами с плавающей точкой. Разработанные программы должны принимать числа в допустимом диапазоне. Например, нужно учитывать области определения и допустимых значений, если это связано с условием задачи.

\begin{enumerate}
    \item Разработать программу, вычисляющую с помощью степенного ряда с точностью не хуже 0,05\% значение функции $\sqrt{1+x}$ для  заданного параметра \textit{x}.
    \item Разработать программу, вычисляющую с помощью степенного ряда с точностью не хуже 0,1\%  значение функции гиперболического синуса $\sh{(x)} = (e^x-e^{-x})/2$ для  заданного параметра \textit{x}.
    \item Разработать программу, вычисляющую с помощью степенного ряда с точностью не хуже 0,1\% значение функции $\cos{(x)}$ для  заданного параметра \textit{x}.
    \item Разработать программу, вычисляющую с помощью степенного ряда с точностью не хуже 0,1\% значение биномиальной функции $(1+x)^{m}$ для конкретных параметров \textit{m} и \textit{x}.
    \item Разработать программу, вычисляющую с помощью степенного ряда с точностью не хуже 0,05\%  значение функции $\arcsin{(x)}$ для  заданного параметра \textit{x}.
    \item Разработать программу, вычисляющую с помощью степенного ряда с точностью не хуже 0,05\%  значение функции $1/e^x$ для  заданного параметра \textit{x}.
    \item Разработать программу, вычисляющую с помощью степенного ряда с точностью не хуже 0,05\% значение функции $\sin{(x)}$ для  заданного параметра \textit{x}.
    \item Разработать программу, вычисляющую с помощью степенного ряда с точностью не хуже 0,1\% значение функции $\arccos(x)$ для  заданного параметра \textit{x}.
    \item Разработать программу, вычисляющую с помощью степенного ряда с точностью не хуже 0,05\% значение функции $\arctan{(x)}$ для  заданного параметра \textit{x}.
	\item Разработать программу, вычисляющую с помощью степенного ряда с точностью не хуже 0,05\% значение функции гиперболического тангенса $\tanh(x)=(e^x-e^{-x})/(e^x+e^{-x})$ для  заданного параметра~\textit{x}.
	\item Разработать программу, вычисляющую с помощью степенного ряда с точностью не хуже 0,05\% значение функции $\textit{1/(1-x)}$ для  заданного параметра \textit{x}.
	\item Разработать программу, вычисляющую с помощью степенного ряда с точностью не хуже 0,05\% значение функции $\tan(x)$ для  заданного параметра \textit{x}.
	\item Разработать программу, вычисляющую с помощью степенного ряда с точностью не хуже 0,1\%  значение функции $e^x$ для  заданного параметра \textit{x}.
    \item Разработать программу, вычисляющую с помощью степенного ряда с точностью не хуже 0,1\% значение функции гиперболического котангенса $\cth{(x)}=(e^x+e^{-x})/(e^x-e^{-x})$ для  заданного параметра~\textit{x}.
    \item Разработать программу, вычисляющую с помощью степенного ряда с точностью не хуже 0,05\% значение функции гиперболического косинуса $\ch(x)=(e^x+e^{-x})/2$ для  заданного параметра x.
    \item Разработать программу, вычисляющую с помощью степенного ряда с точностью не хуже 0,05\% значение функции $e^{-x}$ для  заданного параметра \textit{x}.
	\item Разработать программу, вычисляющую с помощью степенного ряда с точностью не хуже 0,1\% значение функции $\ln{(1-x)}$ для  входного параметра \textit{x}.
    \item Разработать программу вычисления корня квадратного по итерационной формуле Герона Александрийского с точностью не хуже 0,05\%.
	\item Разработать программу вычисления корня кубического из заданного числа согласно быстро сходящемуся итерационному алгоритму определения корня \textit{n}-ной степени с точностью не хуже 0,05\%.
    \item Разработать программу вычисления числа $\pi$ с точностью не хуже 0,05\% посредством произведения элементов ряда Виета.
    \item Разработать программу вычисления числа $\pi$ с точностью не хуже 0,05\% посредством ряда Нилаканта.
    \item Разработать программу вычисления числа $\pi$ с точностью не хуже 0,1\% посредством дзета-функции Римана.
    \item Разработать программу вычисления числа $\pi$ с точностью не хуже 0,05\% посредством произведения элементов ряда Валлиса.
    \item Разработать программу, вычисляющую с помощью  ряда Эйлера с точностью не хуже 0,1\% значение числа \textit{e}.
    \item Разработать программу, решающую вопрос о принадлежности заданных 4-х точек одной окружности.
    \item Разработать программу вычисления корня пятой степени согласно быстро сходящемуся итерационному алгоритму определения корня \textit{n}-той степени с точностью не хуже 0,1\%.
    \item Разработать программу интегрирования функции $y=a+b*x^{-2}$ (задаётся двумя числами а,b) в заданном диапазоне (задаётся так же) методом Симпсона (точность вычислений = 0.0001).
    \item Разработать программу численного интегрирования функции $y=a+b*x^4$ (задаётся действительными числами а,b) в определённом диапазоне целых (задаётся так же) методом прямоугольников с избытком (точность вычислений = 0.0001).
    \item Разработать программу численного интегрирования функции $y=a+b*x^{-4}$ (задаётся действительными числами а,b) в определённом диапазоне целых (задаётся так же) методом средних (точность вычислений = 0.0001).
    \item Разработать программу численного интегрирования функции $y=a+b*x^3$ (задаётся действительными числами а,b) в определённом диапазоне целых (задаётся так же) методом трапеций (точность вычислений = 0.0001).
    \item Разработать программу численного интегрирования функции $y=a+b*x^3$ (задаётся действительными числами а,b) в определённом диапазоне целых (задаётся так же) методом прямоугольников с недостатком (точность вычислений = 0.0001).
    \item Разработать программу, определяющую корень уравнения $2^{x^{2}+1}+x^2-4=0$ методом половинного деления с точностью от 0,001 до 0,00000001 в диапазоне [0;1]. Если диапазон некорректен, то подобрать корректный диапазон.
    \item Разработать программу, определяющую корень уравнения $x^3-0.5x^2+0.2x-4=0$ методом половинного деления с точностью  от 0,001 до 0,00000001 в диапазоне [1;3]. Если диапазон некорректен, то подобрать корректный диапазон.
    \item Разработать программу, определяющую корень уравнения $x^4+2x^3-x-1=0$ методом половинного деления с точностью = от 0,001 до 0,00000001 в диапазоне [0;1]. Если диапазон некорректен, то подобрать корректный диапазон.
    \item Разработать программу, определяющую корень уравнения $x^4-x^3-2.5=0$ методом хорд с точностью от 0,001 до 0,00000001 в диапазоне [1;2]. Если диапазон некорректен, то подобрать корректный диапазон.
    \item Разработать программу, определяющую корень уравнения $2^{x^{2}+1}+x-3=0$ методом хорд с точностью от 0,001 до 0,00000001 в диапазоне [2;3]. Если диапазон некорректен, то подобрать корректный диапазон.
    \item Разработать программу, определяющую корень уравнения $x^5-x-0.2=0$ методом хорд с точностью от 0,001 до 0,00000001 в диапазоне [1;1.1]. Если диапазон некорректен, то подобрать корректный диапазон.
\end{enumerate}
