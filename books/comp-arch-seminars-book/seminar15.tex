\chapter[15]{Семинар 15. Обработка исключений}

Целью семинара является разбора выполненного задания. Обсуждение основных недостатков и их учет при выполнении следующих заданий.

Примерный перечень вопросов:
\begin{enumerate}
    \item Ручная проверка, это не проверка олимпиадных задач. Идет просмотре исходных и компилируемых текстов (после рефакторинга), запуск вручную. Поэтому необходимы элементарные юзабилити и устранение <<запаха>>.
    \item Проверки на корректность командной строки. Наличие аргументов. Наличие нужных файлов.
    \item Проверка диапазонов вводимых массивов.
    \item Хорошо бы при написании ассемблерного кода использовать символические обозначения смещений для локальных переменных.
    \item Ответы на вопросы и претензии.
\end{enumerate}

Все это в режиме импровизации.
