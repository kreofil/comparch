\chapter[17]{Семинар 17. Таймер. Прерывания по таймеру}

Целью семинара является изучение базовых механизмов синхронизации POSIX Threads, обеспечивающих совместное безконфликтное использование ресурсов. Необходимо провести анализ примеров, демонстрирующих различные методы синхронизации и того, что может получиться при отсутствии синхронизации.

Примерный перечень вопросов:
\begin{enumerate}
    \item Пример 01. Перемножение матриц.  Лучше использовать пример 07 (07. Перемножение матриц. Добавление мьютексов для синхронизации очереди вывода данных), так как там имеется больше информации для вывода данных. Первый вариант связан с тем, что можно после запуска корректной программы закомментировать мьютексы и сопоставить результаты. Возможно, из-за того, что вычислительная нагрузка небольшая и потоки отрабатывают быстро, весь вывод данных и формирование матрицы с результатами на выводе произойдут корректно. Можно запустить программу без мьютексов нескольно раз. Тогда, раскомментировав мьютексы, можно увеличить матрицы, например до 10*10. Создать десяток потоков и повторить еще раз. После этого раскомментировать и посмотреть, произойдет ли изменение вывода. То есть, разобраться с тем, что не всегда очевидно сразу, что синхронизация может понадобиться.
    \item Пример 02 (08. Задача о кольцевом буфере. Использование семафоров для синхронизации потоков). Здесь время не играет определенной роли. Поэтому после демонстрации протокола работающей программы можно посмотреть, что будет если закомментировать по очереди семафоры и мьютексы. Помимо этого можно изменить поведени системы. Например, ускорить писателей и читателей. Или резко увеличить тех или других без изменения временных интервалов. В этих ситуациях интересно посмотреть ожидание, когда буфер будет либо полон, либо пуст.
    \item Пример 03 (09. Задача о кольцевом буфере. Использование условных переменных для синхронизации потоков). Но с другими синхропримитивами. Просто акцентировать на специфике и отличии условных переменных от семафоров. Но так же можно поиграться.
    \item Пример 04 (10. Читатели-писатели с общим одномерным массивом. Использование блокировок).Если успеваем, то можно зацепить этот и следующий. Или перенести на следующий семинар, увязав на нем с OpenMP. Здесь, наверное, можно поиграться только с разными интенсивностями работы читателей и писателей. Тем более, что читатели ничего не меняют. Можно поиграться числом разных потоков или интенсивностью их обращения.
    \item Пример 05 (11. Использование барьеров для синхронизации данных). Пример интересен именно тем, где и как убирать или ставить барьеры. Именно они влияют на синхронизацию двух разделенных массивов. Вариант одного барьера закомментирован. С двумя другими можно тоже поиграться, посмотрев разные варианты. Вплоть до отсутствия барьеров. Мьютексы отвечают за вывод. Поэтому их трогать смысла особого нет. Но тоже можно посмотреть, если будет время.
\end{enumerate}

Во всех альтернативных программах можно использовать код как на Си, так и на плюсах. Они работают одинаково.

Пусть студенты сами сделают прогоны и сформируют результаты.

\section{Домашнее задание}

Перенести задание с предыдущего занятия. Пусть проведут дополнительный анализ на необходимость использования синхронизации.

Что было на предыдущем:

Сделать выслать письменный реферат, в котором оценить возможность параллельного выполнения задания №1 на обработку целочисленных одномерных массивов и использованием методов синхронизации. По своему сделанному варианту. Программу писать не нужно.
