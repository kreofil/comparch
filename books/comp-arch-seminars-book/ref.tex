\begin{thebibliography}{00}
%\addcontentsline{toc}{section}{Литература}

\bibitem
{risc-v}
RISC-V International. Описание архитектуры и ее обоснование.
--- \url{https://riscv.org/}

\bibitem
{Harris}
Сара Л. Харрис, Дэвид Харрис.
Цифровая схемотехника и архитектура компьютера: RISC-V / пер. с англ. В. С. Яценкова, А. Ю. Романова; под ред. А. Ю. Романова. --- М.: ДМК Пресс, 2021. --- 810 с.

\bibitem
{Borin}
Edson Borin
An Introduction to Assembly Programming with RISC-V /
Document version: May 9, 2022
--- \url{https://riscv-programming.org/}


\bibitem
{kur-2022}
Курячий Георгий. Архитектура и язык ассемблера RISC-V. Весна 2022.
--- \url{http://uneex.ru/LecturesCMC/ArchitectureAssembler2022}

\bibitem
{aps-git}
Семестровый забег "Архитектур процессорных систем"
--- \url{https://github.com/MPSU/APS}

\bibitem
{kur-youtube-2022}
[UNИX] Архитектура и язык ассемблера RISC-V. Видео лекции. Весна 2022.
--- \url{https://www.youtube.com/playlist?list=PL6kSdcHYB3x6cjkby4H1RuRMzfbEGSNBi}

\bibitem
{aps-youtube}
Архитектуры процессорных систем
--- \url{https://www.youtube.com/c/%D0%90%D0%9F%D0%A1%D0%9F%D0%BE%D0%BF%D0%BE%D0%B2}

\bibitem
{RARS}
RARS -- RISC-V Assembler and Runtime Simulator
--- \url{https://github.com/TheThirdOne/rars}

\bibitem
{Ripes}
Ripes. A visual computer architecture simulator and assembly code editor.
--- \url{https://github.com/mortbopet/Ripes}

\bibitem
{QtRvSim}
QtRvSim–RISC-V CPU simulator for education
--- \url{https://github.com/cvut/qtrvsim}

\bibitem
{Goossens}
Goossens Bernard.
Guide to Computer Processor Architecture. A RISC-V Approach, with High-Level Synthesis.
--- Springer Nature. Switzerland AG --- 2023.

\bibitem
{Booch92}
Буч Г.
Объектно-ориентированное проектирование с примерами применения. /Пер. с англ.
--- М.: Конкорд, 1992. --- 519 с.

\bibitem
{Gay}
Gay Warren.
RISC-V Assembly Language Programming. Using ESP32-C3 and QEMU.
--- Elektor International Media B.V. --- 2022.

\bibitem
{Booch98}
Буч Г.
Объектно-ориентированный анализ и проектирование с примерами приложений на C++, 2-е изд./Пер. с англ.
--- М.: «Издательства Бином», СПб: «Невский диалект», 1998 г. --- 560 с., ил.

\bibitem
{ERD-dict}
Англо-русско-немецко-французский толковый словарь по вычислительной технике и обработке данных, 4132 термина.
Под. ред. А.А. Дородницына. М.: 1978. --- 416 с.

\bibitem
{Gagarina}
Гагарина Л.Г., Кононова А.И.
Архитектура вычислительных систем и Ассемблер с приложением методических указаний к лабораторным работам. Учебное пособие.
--- М.: СОЛОН-Пресс, 2019. --- 368 с.

\bibitem
{elf64}
Формат файла ELF64.
--- \url{https://uclibc.org/docs/elf-64-gen.pdf}

\bibitem
{Plantz}
Plantz Robert G.
Introduction to Computer Organization.
--- 2022

\bibitem
{gdb-stollman}
Ричард Столмен, Роланд Пеш, Стан Шебс и др.
Отладка с помощью GDB.
--- 2000

\bibitem
{gdb-zeller}
Андреас Целлер
Почему не работают программы.
--- М.: Эксмо, 2011. --- 560 с.

\bibitem
{gdb-dive-into-systems}
Suzanne J. Matthews, Tia Newhall, Kevin C. Webb.
Dive into Systems.
--- 2022

\bibitem
{round-up}
Округление. Статья в Википедии.
--- \url{https://ru.wikipedia.org/wiki/%D0%9E%D0%BA%D1%80%D1%83%D0%B3%D0%BB%D0%B5%D0%BD%D0%B8%D0%B5}

\bibitem
{c-lang}
Прохоренок Н.А.
Язык С. Самое необходимое.
--- СПб.: БХВ-Петербург, 2020. --- 480 с.

\bibitem
{nasm-x64}
Йо Ван Гуй.
Программирование на ассемблере x64: от начального уровня до профессионального использования AVX.
--- М.: ДМК Пресс, 2021. --- 332 с.

\bibitem
{wsl}
Установка Linux на Windows с помощью WSL
--- \url{https://docs.microsoft.com/ru-ru/windows/wsl/install}

\bibitem
{vir-box}
Установка Linux на Virtualbox.
--- \url{https://losst.ru/ustanovka-linux-na-virtualbox}

\bibitem
{System-V}
System V Application Binary Interface AMD64 Architecture Processor Supplement (With LP64 and ILP32 Programming Models). Version 1.0. --- 2018.
--- \url{https://github.com/hjl-tools/x86-psABI/wiki/x86-64-psABI-1.0.pdf}



\end{thebibliography}
