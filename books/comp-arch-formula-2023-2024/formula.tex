\documentclass[a4paper, 12pt, oneside]{article}
\usepackage[utf8]{inputenc}
\usepackage[english,russian]{babel}
\usepackage[unicode=true] {hyperref}
\begin{document}

%\maketitle
\begin{center}
\section*{Архитектура вычислительных систем. Формула оценки}
\end{center}

\section{Общие параметры}

Оценка прохождения дисциплины формируется по результатам:
\begin{itemize}
    \item участия в семинарских занятиях;
    \item выполнения домашних заданий
    \item выполнения индивидуальных заданий;
    \item сдачи экзамена.
\end{itemize}

\subsection{Участие в семинарских занятиях}

Каждое семинарское занятие оценивается следующим образом.

\begin{itemize}
    \item отсутствие на семинарском занятии: 0 баллов;
    \item присутствие на занятии и выполнение заданий преподавателя: до 10 баллов;
    \item возможно снижение оценки если не выполняются требования, предъявляемые во время проведения семинара.
\end{itemize}

Общая оценка за все семинары формируется как среднее арифметическое значение без округления по всем прошедшим занятиям:

\begin{verbatim}
    seminars_estimation =
            sum(seminar_estimation[i]) / seminars_number
\end{verbatim}
где: \verb|seminars_estimation| --- усредненная оценка за семинары,\\ \verb|seminar_estimation[i]| --- оценка за \verb|i|-е семинарское занятие,\\ \verb|seminars_number| --- количество проведенных семинарских занятий,\\
\verb|sum| --- сумма всех оценок за семинарские занятия.\\
При учете в общей оценке усредненная оценка за семинары не округляется.

\textbf{При усредненной оценке, не превышающей 4-х баллов, она является блокирующей для получения экзамена автоматом.}

\subsection{Выполнение домашних заданий}
Домашние задания выдаются после проводимых семинаров. Каждое задание оценивается отдельно по десятибальной шкале. Общая оценка за все домашние задания формируется как среднее арифметическое значение без округления:

\begin{verbatim}
    homework_estimation =
            sum(homework_estimation[i]) / homework_number
\end{verbatim}
где: \verb|homework_estimation| --- усредненная оценка за домашние задания,\\ \verb|homework_estimation[i]| --- оценка за \verb|i|-е домашнее задание,\\ \verb|homework_number| --- количество выданных домашних заданий,\\
\verb|sum| --- сумма всех оценок за домашние задания.\\
При учете в общей оценке усредненная оценка за домашние задания не округляется.

\textbf{При усредненной оценке, не превышающей 4-х баллов, она является блокирующей для получения экзамена автоматом.}

\subsection{Выполнение индивидуальных заданий}

В ходе изучения дисциплины предлагается выполнение 4-х индивидуальных заданий.
Каждое из заданий имеет одинаковую весовую оценку. Поэтому общая оценка за все задания формируется как среднее арифметическое значение оценок по всем заданиям.

\begin{verbatim}
    tasks_estimation =
        0.2 * task_estimation1  +
        0.25 * task_estimation2 +
        0.2 * task_estimation3  +
        0.35 * task_estimation4
\end{verbatim}
где:  \verb|tasks_estimation| --- усредненная оценка за индивидуальные задания, \\
\verb|task_estimation1 ... task_estimation4| --- оценка за каждое из заданий.

\textbf{Оценка ниже 4-х баллов хотя бы за одно из индивидуальных заданий является блокирующей для получения экзамена автоматом.}

\subsection{Экзамен}

Допуск к экзамену осуществляется при общей положительной оценке, получаемой с учетом семинарских занятий, домашних заданий и индивидуальных заданий, приведенных к единичному коэффициенту:

\begin{verbatim}
    access_estimation = (
        0.12 * seminars_estimation +
        0.08 * homework_estimation +
        0.6 * tasks_estimation
    ) / 0.8
\end{verbatim}
где \verb|access_estimation| — неокругленная оценка, превышающая 4 балла и определяющая допуск к сдаче экзамена. В противном случае оценка является блокирующей до пересдачи заданий выполненных на отрицательную оценку. Пересдача индивидуальных заданий осуществляется по завершению сессии до проведения переэкзаменовки в соответствии с регламентом. Пересдаются задания, имеющие отрицательную оценку или же те задания, которые имеют минимальную положительную оценку.

Экзамен оценивается по 10 бальной шкале.

\subsection{Итоговая оценка}

Итоговая оценка после проведения экзамена определяется по следующей формуле:

\begin{verbatim}
    result_estimation = math_round (
        0.12 * seminars_estimation +
        0.08 * homework_estimation +
        0.6 * tasks_estimation     +
        0.2 * exam_estimation
    )
\end{verbatim}
где: \verb|result_estimation| --- итоговая оценка,\\
\verb|math_round| --- математическое округление оценки \\ (https://ru.wikipedia.org/wiki/Округление), \\
\verb|exam_estimation| --- оценка за экзамен.

\subsection{Итоговая оценка автоматом}

Получение итоговой оценки автоматом возможно при наличии усредненных положительных оценок (4 балла и выше) за семинарские занятия и домашние задания, а также при наличии положительных оценок за каждое из выполненных заданий. Итоговая оценка при этом рассчитывается с учетом приведения к единице:

\begin{verbatim}
    result_estimation = math_round (
      (
          0.12 * seminars_estimation +
          0.08 * homework_estimation +
          0.6 * tasks_estimation
      ) / 0.8
    )
\end{verbatim}

Возможно также получение оценки автоматом при отсутствии усредненных положительных оценок (4 балла и выше) за семинарские занятия или домашние задания, а также при отсутсвии положительных оценок за выполненные заданий, если окончательная оценка, вычисленная по формуле:
\begin{verbatim}
    low_estimation = (
        0.12 * seminars_estimation +
        0.08 * homework_estimation +
        0.6 * tasks_estimation
    )
\end{verbatim}
без округления превышает 4 балла. В этом случае оценка автоматом выставляется с учетом математического округления:
\begin{verbatim}
    result_low_estimation = math_round(low_estimation)
\end{verbatim}

\textbf{При несогласии с итоговой оценкой автоматом можно ее изменить путем сдачи экзамена.  При этом оценка автоматом аннулируется.}

\end{document}

